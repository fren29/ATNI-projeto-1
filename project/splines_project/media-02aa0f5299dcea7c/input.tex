% Options for packages loaded elsewhere
\PassOptionsToPackage{unicode}{hyperref}
\PassOptionsToPackage{hyphens}{url}
\documentclass[
]{article}
\usepackage{xcolor}
\usepackage{amsmath,amssymb}
\setcounter{secnumdepth}{5}
\usepackage{iftex}
\ifPDFTeX
  \usepackage[T1]{fontenc}
  \usepackage[utf8]{inputenc}
  \usepackage{textcomp} % provide euro and other symbols
\else % if luatex or xetex
  \usepackage{unicode-math} % this also loads fontspec
  \defaultfontfeatures{Scale=MatchLowercase}
  \defaultfontfeatures[\rmfamily]{Ligatures=TeX,Scale=1}
\fi
\usepackage{lmodern}
\ifPDFTeX\else
  % xetex/luatex font selection
\fi
% Use upquote if available, for straight quotes in verbatim environments
\IfFileExists{upquote.sty}{\usepackage{upquote}}{}
\IfFileExists{microtype.sty}{% use microtype if available
  \usepackage[]{microtype}
  \UseMicrotypeSet[protrusion]{basicmath} % disable protrusion for tt fonts
}{}
\makeatletter
\@ifundefined{KOMAClassName}{% if non-KOMA class
  \IfFileExists{parskip.sty}{%
    \usepackage{parskip}
  }{% else
    \setlength{\parindent}{0pt}
    \setlength{\parskip}{6pt plus 2pt minus 1pt}}
}{% if KOMA class
  \KOMAoptions{parskip=half}}
\makeatother
\usepackage{longtable,booktabs,array}
\newcounter{none} % for unnumbered tables
\usepackage{calc} % for calculating minipage widths
% Correct order of tables after \paragraph or \subparagraph
\usepackage{etoolbox}
\makeatletter
\patchcmd\longtable{\par}{\if@noskipsec\mbox{}\fi\par}{}{}
\makeatother
% Allow footnotes in longtable head/foot
\IfFileExists{footnotehyper.sty}{\usepackage{footnotehyper}}{\usepackage{footnote}}
\makesavenoteenv{longtable}
\setlength{\emergencystretch}{3em} % prevent overfull lines
\providecommand{\tightlist}{%
  \setlength{\itemsep}{0pt}\setlength{\parskip}{0pt}}
\usepackage{bookmark}
\IfFileExists{xurl.sty}{\usepackage{xurl}}{} % add URL line breaks if available
\urlstyle{same}
\hypersetup{
  hidelinks,
  pdfcreator={LaTeX via pandoc}}

\author{}
\date{}

\begin{document}

{
\setcounter{tocdepth}{3}
\tableofcontents
}
\section{Projeto 1 — Aproximação Teórica e Numérica
I}\label{projeto-1-aproximauxe7uxe3o-teuxf3rica-e-numuxe9rica-i}

\subsubsection{Estudo sobre a Convergência de Splines Cúbicos
Interpoladores}\label{estudo-sobre-a-converguxeancia-de-splines-cuxfabicos-interpoladores}

\textbf{Autor:} Rodrigo Fassa et al.\\
\textbf{Orientador:} Prof.~André Pierro de Camargo\\
\textbf{Data:} 09/11/2025

\begin{center}\rule{0.5\linewidth}{0.5pt}\end{center}

\subsection{1. Introdução}\label{introduuxe7uxe3o}

Este relatório apresenta o estudo numérico da convergência de
\emph{splines cúbicos interpoladores}, com o objetivo de verificar
experimentalmente a ordem de convergência teórica prevista para o
método. O spline cúbico é uma função polinomial por partes, de classe
(C\^{}2), construída de modo que a curvatura (segunda derivada) varie
suavemente, minimizando a energia elástica da curva.

A teoria garante que, se (f \in C\^{}4{[}a,b{]}), então o erro máximo
satisfaz: {[} E\_n = \max\_\{x \in [a,b]\} \textbar f(x) -
S(x)\textbar{} \approx C \cdot h\^{}4, {]} onde (h) é o espaçamento da
malha.

\begin{center}\rule{0.5\linewidth}{0.5pt}\end{center}

\subsection{2. Metodologia}\label{metodologia}

As rotinas foram implementadas em Python de acordo com o pseudocódigo do
enunciado. As principais funções são:

{\def\LTcaptype{none} % do not increment counter
\begin{longtable}[]{@{}
  >{\raggedright\arraybackslash}p{(\linewidth - 4\tabcolsep) * \real{0.2759}}
  >{\raggedright\arraybackslash}p{(\linewidth - 4\tabcolsep) * \real{0.3103}}
  >{\raggedright\arraybackslash}p{(\linewidth - 4\tabcolsep) * \real{0.4138}}@{}}
\toprule\noalign{}
\begin{minipage}[b]{\linewidth}\raggedright
Módulo
\end{minipage} & \begin{minipage}[b]{\linewidth}\raggedright
Função
\end{minipage} & \begin{minipage}[b]{\linewidth}\raggedright
Descrição
\end{minipage} \\
\midrule\noalign{}
\endhead
\bottomrule\noalign{}
\endlastfoot
\texttt{spline.py} & \texttt{build\_tridiagonal\_system} & Monta o
sistema (T\cdot M=d) para o spline cúbico. \\
\texttt{gauss.py} & \texttt{solve\_by\_gaussian\_elimination} & Resolve
o sistema linear. \\
\texttt{spline.py} & \texttt{compute\_M}, \texttt{compute\_AB},
\texttt{spline\_eval} & Calculam segundas derivadas e coeficientes. \\
\texttt{tarefas.py} & \texttt{tarefa\_convergencia\_*} & Experimentos de
convergência. \\
\texttt{tarefas.py} & \texttt{ajuste\_ordem\_convergencia} & Estima
(\rho) por regressão log–log. \\
\end{longtable}
}

A validação foi feita sobre (f(x)=\cos(x)), em ({[}0, \pi/2{]}), usando
as condições de contorno \textbf{natural} e \textbf{completa}.

\begin{center}\rule{0.5\linewidth}{0.5pt}\end{center}

\subsection{3. Resultados Numéricos}\label{resultados-numuxe9ricos}

\subsubsection{3.1 Spline Natural}\label{spline-natural}

{\def\LTcaptype{none} % do not increment counter
\begin{longtable}[]{@{}rrr@{}}
\toprule\noalign{}
n & h & Eₙ \\
\midrule\noalign{}
\endhead
\bottomrule\noalign{}
\endlastfoot
4 & 0.392699 & 7.725073e-03 \\
8 & 0.196350 & 1.902205e-03 \\
16 & 0.098175 & 4.737284e-04 \\
32 & 0.049087 & 1.183202e-04 \\
64 & 0.024544 & 2.957305e-05 \\
\end{longtable}
}

\textbf{Ordem estimada:} ρ ≈ 2.01

\begin{center}\rule{0.5\linewidth}{0.5pt}\end{center}

\subsubsection{3.2 Spline Completo}\label{spline-completo}

{\def\LTcaptype{none} % do not increment counter
\begin{longtable}[]{@{}rrr@{}}
\toprule\noalign{}
n & h & Eₙ \\
\midrule\noalign{}
\endhead
\bottomrule\noalign{}
\endlastfoot
4 & 0.392699 & 6.324039e-05 \\
8 & 0.196350 & 3.889330e-06 \\
16 & 0.098175 & 2.421787e-07 \\
32 & 0.049087 & 1.512267e-08 \\
64 & 0.024544 & 9.443273e-10 \\
\end{longtable}
}

\textbf{Ordem estimada:} ρ ≈ 4.01

\begin{center}\rule{0.5\linewidth}{0.5pt}\end{center}

\subsubsection{3.3 Gráfico log–log}\label{gruxe1fico-loglog}

\begin{figure}
\centering
{Convergência log–log}
\caption{Convergência log–log}
\end{figure}

O gráfico evidencia uma relação linear entre log(Eₙ) e log(h), com
inclinação próxima de 4.

\begin{center}\rule{0.5\linewidth}{0.5pt}\end{center}

\subsection{4. Discussão e Conclusão}\label{discussuxe3o-e-conclusuxe3o}

Observou-se que o spline natural apresentou erro decaindo
aproximadamente como (E\_n \sim h\^{}2), enquanto o spline completo
atingiu a convergência teórica de quarta ordem ((ρ pprox 4)). A
imposição das derivadas nas extremidades remove o viés de contorno e
garante a suavidade global (C\^{}2).

Assim, o comportamento numérico confirma a teoria apresentada em sala e
conclui que o \emph{spline cúbico completo} é um método de alta precisão
para interpolação suave.

\begin{center}\rule{0.5\linewidth}{0.5pt}\end{center}

\textbf{UFABC — Aproximação Teórica e Numérica I (2025/Q3)}

\end{document}
